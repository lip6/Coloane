\documentclass[a4paper,10pt]{report}


% Title Page
\title{Rapport de travail sur les logs de l'API de Coloane}
\author{Dimitri Charles}


\begin{document}
\maketitle
\section{Introduction}
\indent
Etant donn\'e que Coloane \'echange continuellement des donn\'ees avec FrameKit, l'objectif est de r\'ecup\'erer ces informations et de les tracer dans des fichiers sous diff\'erents formats (XML ou un format simpple). Dans ce cas, le package \textbf{java.util.logging}, pr\'esent depuis le JDK1.4, prend toute son importance. 
\\
Il existe d'autres packages qui sont utilisés pour faire des logs, par exemple le paquet log4j qui se trouve dans \textbf{org.apache.log4j.Logger} et qui n'est pas fourni avec l'API standard de Java. Il est t\'el\'echargeable sur le site d'apache \`a l'adresse \textbf{http://logging.apache.org/site/binindex.cgi}. Le principe est toujours le meme \'à  certaines diff\'erences pr\`es. L'API qui est utilis\'e est le java.util.logging qui est int\'egr\'e au JDK.
\\
\section{Etape pour cr\'eer un logger}
Toute classe qui doit etre trac\'ee doit importer le package \textbf{java.util.logging}. Ce dernier contient un ensemble de m\'ethodes qui permettent de journaliser les \'ev\'enements d'un programme. On commence par cr\'eer un attribut static et prot\'eg\'e  de type Logger qui fait r\'ef\'erence au journal et on appelle la m\'ethode \textbf{getLogger()} pour lui donner un nom. Par convention, le nom du journal est identique au nom complet de la classe :

\begin{center}
 \textbf{protected static Logger logger =
Logger.getLogger("myPackage.mySubPackage.myClasse");}
\end{center}

Vient ensuite la cr\'eation d'un Handler. Un handler est le support sur lequel va afficher nos messages. Il en existe plusieurs types : les handlers de type fichier(FileHandler), socket(SocketHandler), m\'emoire (MemoryHandler),etc. Ici, c'est le FileHandler qui est utils\'e. On l'instancie en passant en le param\`etre le nom du fichier à cr\'eer. Puis, on l'associe au journal via la m\'ethode \textbf{addHandler}.Exemple : 
\begin{center}
 \textbf{Handler fh = new FileHandler("myLog.log");
logger.addHandler(fh);}
\end{center}
Apr\`es la cr\'eation du Handler, il faut d\'efinir un format d'affichage. Il existe 2 types : soit un format simple(classe \textbf{SimpleFormatter}) avec du texte brut ou un format XML(classe \textbf{XMLFormatter}) dont tout est g\'en\'er\'e en XML. Il suffit d'appeler la m\'ethode setFormatter pour associer le format d\'esir\'e au fichier.
Pour journaliser les \'evenements dans notre programme, on fait appel aux m\'ethodes pr\'ed\'efinies dans la classe Logger.  Tout type d'affichage d\'epend d'un niveau de criticit\'e dont le poids est diff\'erent (\textbf{ALL} pour tout afficher, \textbf{Severe} niveau le plus \'elev\'e, critique, \textbf{Warning} pour un avertissement, \textbf{Info} pour donner un information, le reste \textbf{CONFIG, FINE, FINER, FINEST} pour les niveaux faibles et \textbf{OFF} pour ne rien afficher). Le niveau utilis\'e ci est \textbf{ALL}.


\section{Utilisation dans l'API de Coloane}
\indent
On a suivi cette meme d\'emarche pour tracer tout ce qui se passe dans l'API de communication de Coloane avec quelques diff\'érences. Au lieu de d\'eclarer un membre \textbf{\underline{static}} et \textbf{\underline{prot\'eg\'e}} dans la classe principale API, on d\'eclare un membre static et public afin qu'il soit visible par les autres classes des autres packages. Un import suffira pour appeler ce membre dans les autres classes. Le niveau d'affichage est \textbf{ALL}, ce qui signifie que tout sera repertori\'e : les entr\'es dans les m\'ethodes, leurs param\'etres, leur valeurs de retour quand il y en a et aussi les exceptions qui sont lev\'ees, les messages d'informations (\textit{Sys.out.println()}).
\\

\section{Conclusion}
\indent
Dans ce rapport, on a bri\`evement pr\'esent\'e le package \textbf{java.util.logging} et son utilisation dans Coloane. Il serait de d\'efinir notre propre format d'affichage puisqu'il a \'et\'e propos\'e d'\'ecrire des scripts afin de traiter les informations obtenues dans les fichiers de logs. Dans ce cas, Il suffit d'\'etendre la classe Formatter pr\'esent dans le package. Vu l'abondance d'informations que l'aura avec le niveau \textbf{ALL} du programme, il serait int\'eressant de d\'efinir un niveau de trace pour chaque type de personne utilisant le programme.
\end{document}          
