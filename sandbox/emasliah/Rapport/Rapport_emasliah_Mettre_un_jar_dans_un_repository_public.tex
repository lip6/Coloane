\documentclass[a4paper,10pt]{article}

\usepackage[utf-8]{inputenc}

%opening
\title{Mettre un jar dans un repository public}
\author{Eric Masliah\\
Equipe MoVe}
\date{17 Aout 2007}

\begin{document}

\maketitle

\begin{abstract}
Ce document donne les différentes étapes pour déployer un jar.
\end{abstract}

\section{Introduction}
Un projet maven fonctionne avec des dépendances, qu'il trouve soit dans son repository local, soit dans un repository public. Nous allons voir ici comment rajouter une dépendance de notre projet dans un repository public.

\section{Maven Deploy Plugin}
Nous allons utiliser le Maven Deploy Plugin\\ (http://maven.apache.org/plugins/maven-deploy-plugin/index.html). Plus precicement, nous allons utiliser la buts deploy:deploy-file. Nous allons considérer que nous sommes sur la machine sur laquelle on veut deployer le jar, que nous voulons le mettre dans le repository appeller m2 et que nous somme dans le repertoire parent de ce dossier.

\section{Deploy-File}
Soit notre fichier jar ainsi : org.eclipse.emf.converter\_2.2.1.v200609071016.jar.\\
On peut considerer que le groupId = org.eclipse.emf, l'artifactId = converter, la version = 2.2.1.v200609071016 et le packaging = jar. Il y a donc juste a taper : \\
\begin{verbatim}
mvn deploy:deploy-file -DgroupId=org.eclipse.emf \
                       -DartifactId=converter \
                       -Dversion=2.2.1.v200609071016 \
                       -Dfile=filepath/org.eclipse.emf.converter_2.2.1.v200609071016.jar \
                       -Dpackaging=jar \
                       -Durl=file://m2
\end{verbatim}

\section{Que mettre dans le pom ?}
Les projets voulant utiliser cette dependance devrons tout betement mettre dans le pom :
\begin{verbatim}
<dependency>
  <groupId>org.eclipse.emf</groupId>
  <artifactId>converter</artifactId>
  <version>2.2.1.v200609071016</version>
</dependency>
\end{verbatim}
Et avoir le repository dans leur pom.

\section{Conclusion}
Et voila, rien de plus simple !

\end{document}
