\documentclass[a4paper,10pt]{article}

\usepackage[utf-8]{inputenc}

%opening
\title{Rapport : Continuum}
\author{Eric Masliah\\
Equipe MoVe}
\date{22 Aout 2007}

\begin{document}

\maketitle

\begin{abstract}
Ce document explique le fonctionnement principal de Continuum relié avec un projet Maven 2.
\end{abstract}

\section{Introduction}
Continuum est une interface web pour les projects Maven 2. J'utilise la version 1.1-beta-2.

\section{Ajouter un project}
Cliquer sur ``Maven 2.0.x Project'' à gauche.\\
Mettre l'adresse du pom. Le pom doit contenir, en plus des informations obligatoire, les information du scm car Continuum va telecharger de project.\\
\textit{Ex : }
\begin{verbatim}
<scm>
  <developerConnection>
     scm:svn:https://srcdev.lip6.fr/svn/research/${project.artifactId}/trunk
  <developerConnection>
</scm>
\end{verbatim}

\section{Fonctionnalités}
Continuum à plusieurs interet, outre l'interface web.

\subsection{Schedules}
Grace à un system d'horloge qui ne sont appelées que lorsque que le svn change, la derniére version des projects est toujours testée

\subsection{Release}
Grace au bouton release, on peut faire une release d'un project facilement et avec une belle interface graphique.

\subsection{Notifiers}
Continuum peux envoyer des mails sur divers évenements, lorsqu'un project ne build pas par exemple.

\section{Conclusion}
Encore en version béta, quelques fonctionnalités ne sont pas encore parfaite, Continuum marche trés bien avec un project Maven 2 sous svn.

\end{document}
