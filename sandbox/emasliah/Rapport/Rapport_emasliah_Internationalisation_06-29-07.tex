\documentclass[a4paper,10pt]{article}

\usepackage[utf-8]{inputenc}

%opening
\title{Rapport : Internationalisation}
\author{Eric Masliah\\
Equipe MoVe}
\date{29 Juin 2007}

\begin{document}

\maketitle

\begin{abstract}
Ce document décrit la mise en place de l'internationalisation dans un projet java
\end{abstract}


\section{Introduction}
L'internationalisation d'un programme est trés interrésante car cela permet une plus grande distribution de ce dernier. Voyons comment cela s'implémente en java.

\section{Comment faire ?}
En java, l'internationalisation se faire trés facilement. Il suffit de creer deux objets :
\begin{itemize}
\item[-] un objet Locale à qui l'on passera la langue et le pays/region désirés,
\item[-] un objet ResourceBundle a qui l'on passera le nom générique du fichier ou se trouve la traduction ainsi que l'objet Locale,
\end{itemize}
puis d'appeller la methode getString du ResourceBundle en lui passant l'indentifiant du texte recherché.

\section{Le fichier .properties}
La traduction se trouve dans un fichier.properties

\subsection{Nom}
Le fichier ou se trouve la traduction se nomme 
\texttt{nom\_générique\_du\_fichier}\_\textit{langue}\_\textit{pays}.properties. Le fichier de traduction par défault se nomme \texttt{nom\_générique\_du\_fichier}.properties.

\subsection{Contenu}
Le fichier .properties est de la forme suivante :\\
identifiant\_mis\_dans\_le\_fichier\_java = traduction\\
La traduction dois être êcrite en utf-16, soit tous les caractéres latin non accentués peuvent être écrit normalement et tout le reste doit être ecrit avec son code utf-16, soit :
\begin{verbatim}
 exemple.emasliah.1 = Voici de l'arabe : \u062E\u0646
\end{verbatim}
%Ce qui affichera :\\
% Voici de l'arabe : خن 
On peux trouver un convertisseur de caractère sur cette page :  http://hapax.qc.ca/conversion.fr.html

\section{Conclusion}
Cette technique est facile à mettre en place mais il faut faire attention au probléme de longeur de la traduction, de l'encodage, etc.

\end{document}
