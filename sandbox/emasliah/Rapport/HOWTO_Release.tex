\documentclass[a4paper,10pt]{article}

\usepackage[utf-8]{inputenc}

%opening
\title{Comment faire une release}
\author{Eric Masliah\\
Equipe MoVe}
\date{20 Aout 2007}

\begin{document}

\maketitle

\begin{abstract}
Ce document donne les différentes étapes pour faire une release d'un projet Maven 2  avec Continuum version 1.1-beta-1.
\end{abstract}

\section{Introduction}
La regle principale pour faire une release est que le projet ne dois pas avoir de dépendances SNAPSHOT. Si c'est un project multi-module, ses modules peuvent en avoir si et seulement si celles-ci sont vers d'autres modules. Une release se deroule en 2 etapes : prepare et perform. La premiere permet de tagger le trunk, la deuxieme de mettre les jars sur un serveur distant.

\section{Release:Prepare}
Cliquer sur le bouton ``Release'' de Continuum (ou l'icone Release)\\
Cocher ``Prepare project for release''.\\
Cliquer sur le bouton ``Submit''.\\
Remplir la case ``SCM Password'' avec le mot de passe corespondant au ``SCM Username''.\\
Remplir la case ``SCM Tag'' avec le nom du dossier ou sera mis la version taggée, soit nom\_du\_projet-version\_de\_la\_release.textit{Ex : coloane-1.3.0}\\
Attendre la fin de ``Executing Release Goal''.\\
Cliquer sur le bouton ``Done''.

\section{Release:Perform}
Cocher ``Perform project release''.\\
Cliquer sur le bouton ``Submit''.\\
Cliquer sur le bouton ``Submit''.\\
Attendre la fin de ``Executing Release Goal''.\\
Cliquer sur le bouton ``Done''.

\section{Conclusion}
Vous avez maintenant une version taggée sur le svn et les jars sur un repository distant. Vous avez aussi la release installer dans le repository local. De plus vous avez une copie de la version taggée dans une branches, il faut changer dans le pom de cette derniere l'adresse du scm en remplacant tags par branches.

\end{document}
