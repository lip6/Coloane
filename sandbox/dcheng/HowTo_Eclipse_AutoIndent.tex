\documentclass{article}

\usepackage[utf-8]{inputenc}

\title{Eclipse: Java code style}
\author{David Cheng}
\date{11 Juillet 2007}

\begin{document}
\maketitle
\section{Objectif}
L'intéret d'intégrer son propre code style dans un projet Eclipse 
est de pouvoir avoir la même indentation pour tous les développeurs 
grâce à l'indentation automatique d'Eclipse.

\section{Installation d'un code style}
Comment installer un code style existant pour eclipe:\\

Dans Eclipse aller dans  "Window-$>$ Preferences",
choisir "Java" puis ``Code Style'' et enfin, ``Formatter''.
Cliquer sur ``Import`` et choisir le fichier *.xml.
Choisir un nom puis appliquer les modifications.\\

Votre code peut maintenant être indenté automatiquement tout en suivant les règles du code style.

\section{Cr\'eation}

Comment cr\'eer une configuration code style pour Eclipse:\\

Aller dans "Window -$>$ Preferences''.
Choisir "Java" puis ``Code Style'' et enfin, ``Formatter'' puis le bouton "New".\\

Saisir un nom (et si souhait\'e, une description) puis terminer en cliquant sur "OK".



\section{Configuration}

Comment configurer un code style:\\
Aller dans "Window -$>$ Preferences''.
Choisir "Java" puis  ``Code Style'' et enfin, ``Formatter'' puis le bouton "Edit".\\

L'onglet "Indentation" nous permet de spécifier le type d'indentation (tabulation ou espace) ainsi que la taille de l'indentation entre chaque ligne.\\

L'onglet "Line Warpping" nous permet de choisir si une expression doit être suivie ou précédée d'un saut de ligne ou non ainsi que le nombre de caractères maximum sur une ligne.

\section{Commande}

La commande permettant d'indenter automatiquement un code est:\\

CTRL+SHIFT+F
\end{document}

