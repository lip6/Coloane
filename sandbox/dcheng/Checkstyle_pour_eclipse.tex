\documentclass{article}

\usepackage[francais]{babel}

\title{Checkstyle sous Eclipse}
\author{David Cheng}
\date{13 Juin 2007}

\begin{document}
\maketitle


\section{Installation}
Comment installer le plugin checkstyle pour eclipe:\\

Dans Eclipse aller dans  "Help-$>$ Software Updates-$>$Find and Install",
choisir "Search for new features to install" puis cliquer sur "Next".  
Cliquer sur "New Remote Site".

Saisir un nom de votre choix puis l'adresse suivante: 
"http://eclipse-cs.sourceforge.net/update"

Avancer dans les diff\'erentes pages pour finaliser l'installation.

\section{Configuration}
\subsection{Cr\'eation}

Comment cr\'eer une configuration checkstyle:\\

Aller dans "Window -$>$ Preferences -$>$ Checkstyle"
Choisir l'onglet "Local Check Configurations"
puis le bouton "New".\\

Saisir un nom (et si souhait\'e, une description) puis terminer en cliquant sur "OK".



\subsection{Import}

Comment importer un fichier de configuration checkstyle d\'ej\`a existant:\\

Aller dans "Window -$>$ Preferences -$>$ Checkstyle"
Choisir l'onglet "Local Check Configurations"
puis le bouton "New".\\

Saisir un nom (et si souhait\'e, une description), cliquer sur le bouton "Import" 
puis selectionner le fichier de configuration *.xml.\\

Pour chaque projet, aller dans "Project-$>$Properties-$>$Checkstyle", 
dans l'onglet Main, choisir le checkstyle que vous venez d'importer.


\subsection{Configuration}

Comment configurer un checkstyle:\\

Aller dans "Window -$>$ Preferences -$>$ Checkstyle" 
puis cliquer sur "Configure".

Dans cette nouvelle fen\^etre, la liste de gauche repr\'esente toutes les contraintes 
possibles qu'un checkstyle peut avoir, \`a droite, les contraintes existantes.

\subsection{Utilisation}

Il existe deux m\'ethodes d'utilisation du checkstyle associ\'e \`a un projet.\\

La premiere consiste \`a constamment v\'erifier le projet.
Il suffit de faire un clic droit sur le projet \`a v\'erifier 
puis choisir l'option "Checkstyle -$>$ Activate Checkstyle".
Ainsi, \`a chaque sauvegarde, le checkstyle se relance et v\'erifie \`a nouveau le code.\\

La seconde consiste \`a v\'erifier manuellement les classes Java.
Apr\`es chaque sauvegarde, il faut faire un clic droit sur la classe \`a v\'erifier 
puis choisir l'option "Checkstyle -$>$ Check code with Checkstyle".


\end{document}

