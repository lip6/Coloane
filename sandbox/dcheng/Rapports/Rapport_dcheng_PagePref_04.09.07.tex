\documentclass{article}

\usepackage[francais]{babel}

\title{Page de pr\'ef\'erences}
\author{David Cheng}
\date{04 Septembre 2007}

\begin{document}
\maketitle

\section{Objectifs}

L'objectif est d'ajouter une base de page de pr\'ef\'erences.
Ainsi, il sera possible \`a l'avenir de param\'etrer Coloane en y ajoutant ses pr\'ef\'erences.



\section{Comment ajouter une categorie de pr\'ef\'erence dans le menu existant d'Eclipe} 

Les modifications du menu de la page de pr\'ef\'erences se font dans "plugin.xml".
Il faut y ajouter de nouvelles balises.

Exemple:
\begin{verbatim}
<extension point="org.eclipse.ui.preferencePages">
        <page id="id du pere"
                name="Nom de la page Pere"
                class="classe contenant l'affichage de la page p\`ere"
        </page>
        <page id="id du fils"
                name="Nom de la page Fils"
                class="classe contenant l'affichage de la page fils"
                category="id du pere">
        </page>
 </extension>
\end{verbatim}

On obtient alors un menu tel qui suit:
\begin{verbatim}
Nom de la page Pere
-----Nom de la page Fils
\end{verbatim}

\section{Cr\'eation de la page de pr\'ef\'erence}

Une page de pr\'ef\'erence doit \'etendre la classe "PreferencePage" et implementer ``IWorkbenchPreferencePage''.
Elle doit d\'efinir les methodes abstraites ``void init(IWorkbench workbench)'' et ``final Control createContents(Composite parent)''.

La m\'ethode ``init'' sert \`a initiliser le ``preferenceStore'' qui permet de stocker les pr\'ef\'erences enregistr\'ees:

\begin{verbatim}
 public final void init(IWorkbench workbench) {
		setPreferenceStore(Coloane.getDefault().getPreferenceStore());
	}
\end{verbatim}

La m\'ethode createContents permet de cr\'eer l'affichage de la page.\\

Il est aussi possible de red\'efinir les methodes permformDefault et performOk qui indiquent les instructions \`a ex\'ecuter lorsque l'on appuie sur les boutons Restore default et OK.
(Il est inutil de red\'efinir performApply qui n'effectue qu'un simple performOk)

\section{Implementation de la page de pr\'ef\'erence}

Afin de pouvoir sauvegarder ses pr\'ef\'erences, il faut commencer par les initialiser dans la classe ``Plugin.java''.

Pour ce faire, il faut y ajouter la m\'ethode "void initializeDefaultPreferences(IPreferenceStore store)".\\

Exemple:
\begin{verbatim}
private static final String Pref1 = "maPref";

protected final void initializeDefaultPreferences(IPreferenceStore store) {
		store.setDefault(Platform.getResourceBundle(Pref1,"");
}
\end{verbatim}

La pr\'ef\'erence Pref1 dont le l'identifiant est ``maPref'' est initialis\'e \`a ``''.

Pour sauvegarder une pr\'ef\'erence, il faut modifier la valeur point\'ee par Pref1.
\begin{verbatim}
 Plugin.getDefault().getPreferenceStore().setValue(Pref1, "nouvelle valeur");
\end{verbatim}

Pour r\'ecuperer une pr\'ef\'erence, il suffit de faire appelle \`a la fonction getString() de PreferenceStore.
\begin{verbatim}
 Plugin.getDefault().getPreferenceStore().getString(Pref1)
\end{verbatim}
Cette instruction revera donc: "nouvelle valeur".


\section{Page de pr\'ef\'erences de Coloane}

La page de pr\'ef\'erence actuelle de Coloane permet d'y sauvegarder son login ainsi que de choisir son serveur par defaut. Il ne sera donc plus n\'ecessaire de les saisir lors de l'authentification.\\

Il est aussi possible de d\'efinir le niveau debug.\\

Une autre fonctionnalit\'e est d'affecter des couleurs personnalis\'ees \`a chaque \'el\'ement du mod\`ele.
En effet, dans le menu Colors, il est possible de choisir une couleur pour les noeuds et les arcs.


\section{Conclusion}

Les possibilit\'es de personnalisation sont grandes gr\^ace aux pages de pr\'ef\'erences.
Cela pourrait simplifier l'utilisation de Coloane en y ajoutant l'automatisation de t\^aches suivant les pr\'ef\'erences.

\end{document}

