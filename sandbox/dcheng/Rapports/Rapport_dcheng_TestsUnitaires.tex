\documentclass{article}

\usepackage[francais]{babel}

\title{Rapport: Tests unitaires}
\author{David Cheng}
\date{06 Septembre 2007}

\begin{document}
\maketitle
\section{Objectif}

L'objectif de cette partie est de faciliter la cr\'eation de nouveaux sc\'enarii de tests unitaires.
Pour ce faire, des m\'ethodes de ``tests'' ont \'et\'e cr\'e\'ees pour chaque \'el\'ement du mod\`ele et autres m\'ethodes test\'ees dans le sc\'enario.


\section{Les classes}
\subsection{Arc}
Il est possible ajouter un arc al\'eatoirement avec addArcOK(model, arc) et addArcNull(model, arc), d'ajouter un arc dont l'identifiant existe d\'ej\`a dans le mod\`ele avec addArcWrongID(model) et retirer un arc al\'eatoirement avec removeArc(model).

\subsection{Noeud}
Il est possible d'ajouter un noeud dont l'identifiant existe d\'ej\`a dans le mod\`ele avec nodeArcWrongID(model) et retirer un arc al\'eatoirement avec removeNode(model).

\subsection{Attribut}
La m\'ethode aleaAttribute(model) ajoute un attribut al\'eatoirement \`a n'importe quel \'el\'ement du mod\`ele ou au mod\`ele lui-m\^eme.


\subsection{PI}
aleaAddPI(arc, int posX, int posY) ajoute un point d'inflexion de coordonn\'es (posX, posY) \`a l'arc en param\`etre.

aleaRemovePI(arc, int posX, int posY) retire un point d'inflexion de coordonn\'es (posX, posY) \`a l'arc en param\`etre, s'il existe.

\subsection{Translate}
compareTranslate(t1, t2) compare t1 et t2, deux vecteurs de String qui sont des traductions de mod\`ele en CAMI.

testTranslate(IModel model) test le translate et le buildModel pour le model ``model''.

\section{Conclusion}
Toutes ces m\'ethodes permettront de recr\'eer de nouveaux sc\'enarii.

\end{document}

