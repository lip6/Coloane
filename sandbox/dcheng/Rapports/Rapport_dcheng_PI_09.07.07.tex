\documentclass{article}

\usepackage[francais]{babel}

\title{M\'eta-mod\`ele - Points interm\'ediaires}
\author{David Cheng}
\date{09 Juillet 2007}

\begin{document}
\maketitle

\section{Objectifs}

L'objectif est d'avoir la possibilit\'e d'ajouter des points d'inflexion aux arcs d'un mod\`ele.
Ainsi, on aura la possibilit\'e de courber les arcs sur la repr\'esentation graphique du mod\`ele sur Coloane.
Il faut \'egalement modifier le package coloane pour que ce changement soit pris en compte lors d'un import/export.



\section{Modifications dans le package interfaces}

Afin de repr\'esenter les points intermediaires, un nouvel objet Position a \'et\'e cr\'e\'e et plac\'e dans le package objects.
La classe Position ne contient que des constructeurs et un getter/setter.\\

La liste des positions intermediaires est repr\'esent\'ee dans la classe Arc par un vecteur de IPosition.

Des m\'ethodes ont \'et\'e cr\'e\'ees pour l'ajout et la suppression de points intermediaires (PI) ainsi que pour la r\'ecup\'eration du n-i\`eme PI.

L'ajout d'un PI dans un arc l\'eve une exception si un PI ayant des positions (x,y) identiques existent d\'ej\`a pour cet arc.

La suppression d'un PI dans un arc l\'eve une exception si ce PI n'existe pour cet arc.

Les m\'ethodes: \\

\begin{itemize}
\item \textit{Vector $<$IPosition$>$ getListOfPi()} :

Retourne la liste des positions intermediaires.\\

\item \textit{void addPI(int x, int y) throws SyntaxErrorException} :

Pour chaque \'el\'ement de la liste des positions intermediaires, teste si un \'el\'ement de coordon\'ee (x,y) existe.
Si oui, une exception est lev\'ee, sinon, l'ajout est effectu\'e.\\

\item  \textit{void removePI(int x, int y) throws SyntaxErrorException} :

Pour chaque \'el\'ement de la liste des positions intermediaires, teste si un \'el\'ement de coordon\'ee (x,y) existe.
Si oui, la suppression est effectu\'ee, sinon, une exception est lev\'ee.\\

\item IPosition getNthPI(int index) :

Renvoie la position intermediaire qui se situe \`a l'index "index".\\
\end{itemize}


\section{Modifications dans le package coloane}

Suite \`a l'ajout d'une liste des points intermediaires pour un arc, il faut modifier la m\'ethode translate() de la classe Arc ainsi que la m\'ethode buildModel() de la classe Model.\\

En effet, les points d'inflexion doivent \^etre pris en compte lors de l'import et de l'export du mod\`ele.\\

Dans la m\'ethode translate(), une boucle sur la liste des positions intermediaires (listOfPI) a \'et\'e ajout\'ee pour r\'ecup\'erer et traduire en CAMI les PI.

Dans la m\'ethode buildModel(), un bloc "if" a \'et\'e ajout\'e afin de lire les PI qui sont au format PI(-1,x,y,-1) et de les ajouter \`a l'arc le plus r\'ecemment visit\'e (dont l'identifiant est stock\'e dans lastArcId).

\section{Tests}

Des tests unitaires ont \'et\'e ajout\'ees au sc\'enario existant.
Une classe de test ,testModel\_PI, a \'et\'e cr\'e\'ee pour regrouper les m\'ethodes de tests propres aux points d'inflexion.

Ces m\'ethodes testent l'ajout et la suppression, correcte ou non, de PI dans un arc.\\

Dans le sc\'enario, tous les 15 tours, on ajoute d'un point d'inflexion, dont les coordonn\'ees sont choisies al\'eatoirement entre 0 et 50, \`a un arc pris au hasard.\\

A chaque ajout d'un nouvel arc, on retire un PI, dont les coordonn\'ees ont choisies al\'eatoirement entre 0  et 50, \`a un arc pris au hasard.

\section{Note}

L'ordre des PI dans la liste des positions intermediaires de l'arc a une importance.\\

Le premier PI de la liste repr\'esente le premier point d'inflexion de l'arc, le dernier PI de la liste repr\'esente le dernier point d'inflexion.

\section{Conclusion}

Les positions interm\'ediaires ajout\'ees au m\'eta-mod\'ele, il ne reste qu'\`a les g\'erer au niveau IHM pour obtenir un mod\'ele avec des arcs coud\'es.
Il faut aussi que ces points soient pris en compte lors de la sauvegarde du mod\'ele.
(En cours d'impl\'ementation)

\end{document}

