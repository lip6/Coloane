\documentclass{article}

\usepackage[francais]{babel}
\usepackage[utf8]{inputenc}

%\usepackage[latin1]{inputenc}

%%\usepackage[T1]{fontenc}
%%\usepackage[french]{babel}

%%%\usepackage[francais]{babel}
%%%\usepackage[T1]{fontenc}
%%%\usepackage[latin1]{inputenc}

%%%%\usepackage[T1]{fontenc}
%%%%\usepackage[frenchb]{babel}

\title{Rapport de stage - Coloane}
\date{Janvier 2008}
\author{Monir CHAOUKI}

\begin{document}

\maketitle

\begin{quotation}
\textit{
Le but de mon stage est, la concéption d'un point d'extension, et la réalisation 
d'une extension pour Coloane. La concéption de points d'extensions, permettra à 
Coloane, d'intégrer de nouvelles fonctionalités, via les contributions 
d'autres développeurs (en respectant certaines conditions), sans que ceux--ci,
 est à modifier le code de Coloane.}
\end{quotation}


{\Huge ATTETION\\ VERSION NOM FINALE\\ AVEC PLEINS DE FAUTES \\NOTEMENT DANS LES PARTIES 4 et 5}


\newpage

\section{Notions impotrtantes}
Nous allons présenter quelques notions, avant de passer à la présentation de 
Coloane, et aux mécanismes mis en jeu pour la conception et à la réalisation 
d'un point d'extension.

\subsection{Plugin}
Un plugin, est un programme qui s'ajoute à une application principale, pour lui 
offrire de nouvelles fonctionnalités. Il existe plusieurs applications qui 
proposent d'ajouter des plugins. Par exemples, le navigateur internet 
"Mozilla FireFox", propose plusieurs plugins allant du simple traducteur, aux 
lecteur de MP3 intégré au navigateur. Tous ces plugins sont issus des 
contributions de developpeurs.

\subsection{Extensions}
Une extension peut être vue comme un plugin limité. En faite, une extension, 
permet d'ajouter de nouveaux services, à une fonctionnalité déjà présente dans 
une application. Reprenons notre précédent exemple sur le navigateur internet 
"Mozilla FireFox". Comme il a été dit plus haut, "Mozilla FireFox" proposent 
des plugins qui permettent de traduire des pages, dans une langue spécifique. 
Il est possible, si les plugins le permettent, d'ajouter de nouvelle langue au 
traducteur, via les contributions d'autres developpeurs, si ceux--ci developpent 
un extensions qui s'ajoute aux plugins existant.

\subsection{Points d'extensions}
Nous avons vue, qu'une extension, est un service qui venai s'ajouter aux 
fonctionnalités, d'une application (ou d'un plugin). Pour qu'une extension 
puisse s'ajoute à une application (ou à un plugin), il faut que celle--ci 
déclare un point d'extension. La déclaration de points d'extensions, permet 
ainssi, à une application d'être extensible, et d'offrire de nouveaux services, 
venant d'autres contributions. Donc, un point d'extension, permet d'indiquer, 
aux extensions où elle doivent venire se greffer, pour qu'elle puisse enrichire 
une application (ou un plugin).

\newpage

\section{Coloane}

\subsection{Presentation de Coloane}
Coloane est un plugin, pour Eclipse, qui permet de modéliser, des reseaux de 
Petri. L'une de ses philosophie est d'être multi--platforme. Le moteur 
physique, utilise un format non standard, le CAMI, pour travailler, sur les 
reseaux de Petri. 

\subsection{Evolutions de Coloane}
L'une des évolutions trés impotrantes, à mettre en place pour Coloane, c'est 
la possiblité d'utilsier plusieur formats, permettant ainssi l'échange de 
différents formats de fichiers.

En effet, aujourd'hui, l'inconveniant majeur de Coloane commme nous l'avons 
écrit plus haut, c'est qu'il n'utilisent qu'un format, non standardiser: 
le CAMI. Or, il existe plusieurs formats pour la représentation des reseaux de 
Petri, dont un principale en cours de normalisation: le PNML.

\subsection{Problémes liés aux l'evolutions de Coloane}
Aujourd'hui, il est biensûre possible de faire évoluer Coloane, mais cela 
implique d'entrer à l'intèrieur du code, et de le modifier directement pour 
ajouter les évolutions voulus.

Or, le fait d'entrer dans le code, de Coloane, qui est actuelement stable, est 
un facteur important d'apparition de bugs. De plus, si plusieurs développeurs 
souhaitent contribuer au projet Coloane pour le faire évoluer, le code risque 
vite d'être confus, sans régles etablies.

Il serait donc, intéresant de pouvoire faire évoluer Coloane, sans toucher 
au code. Le fait de faire évoluer Coloane correspondrait, à ajouter de 
nouvelles fonctionnalitées. Or, ajouter des fonctionnalitées, c'est ajouter un 
plugin ou une extension.

Par conséquant, si l'on veut ajouter des fonctionnalités à Coloane, il faudrait 
songer à utiliser des plugins qui se mettent à coter de lui, ou des extensions 
qui viennent se greffer à lui.

\newpage

\section{Convértiseur de format de représentations de Réseau de Petri}
Nous avons vue précedement, que l'une des évolutions importantes pour 
Coloane, à mettre en place, est un convértiseur de format de représentations de 
Réseau de Petri, permettant ainssi à Coloane d'importer et exporter différents 
formats dont, le plus importants le: PNML.

\subsection{Solutitions envisagées}
Comme nous l'avons écris, deux solutions sont envisagales pour permettre à 
Coloane d'integré de nouvelle fonctionnalités et d'être évolutif (i.e. intégrer 
de nouveaux formats): un plugin et un point d'extension.

\subsubsection{le plugin}
Dans le cas d'un plugin, Coloane, n'aura aucune mettrise au niveau de 
l'intérface graphique qui permet d'intéragir avec l'utlisateur pour 
d'importations et d'exportations de nouveaux format . Toutes les contributions 
pourrons gérer l'intérface graphique,et implementer les fonctions d'importations 
et d'exportations de nouveaux format, comme bon leurs semblent.

\subsubsection{le point d'extension}
Dans le cas d'un point d'extension, Coloane gérera toujours l'intérface 
graphique, mais délèguera aux extensions qui viendront se greffer sur ce point, 
seulement, les fonctions d'importations et d'exportations de nouveaux formats et 
ceux--ci devrons respecter certaines régles.

\subsection{Solutitions choisies}
Il est bien évident que nous allons choisir d'utiliser un point d'extension dans 
Coloane pour que l'on est un certaines maîtrise, des contributions, permettant 
de faire des importations et exportations de nouveaux format.

En effet, chaque contributions devrons avoir la même intérace graphique pour 
interagire avec l'utilisateur, pour que celui--ci, ne soit pas perturber 
lorsqu'il veut exporter ou importer dans des formats diffférent. De plus, on 
imposera aux contributions de respecter certaines régles qui est en faites: 
une intérfaces à implementer. Ainssi toutes nouvelles contributions offrants de 
nouveaux format n'aura qu' à offrire ses service en utilisant le point 
d'extension définie par Coloane.

\newpage

\section{Concéption du point d'extension}
Il faut préparer le plugin Coloane, pour permerttre aux contributions d'offrire 
de nouveaux services, d'importations et d'exportations. Pour cela, il faut 
ajouter quelques lignes dans le fichier 'plugin.xml' pour définire les points 
d'extensions, créer les fichier 'imports.exsd' et 'imports.exsd' dans un 
repertoire 'schema' qui définissent la grammaire des points d'extensions, et 
modifier le fichier 'MANIFEST.MF' pour lui spécifier les packages à exporter et 
qui sont sucéptibles d'être utiliser par les extensions.

Une remarque importante, Eclipse offre une intérface graphique trés agréable : 
PDE, qui permet  la concéption et la definition de points d'extensions sont 
diffuclées.

\subsection{plugin.xml}
Dans le fichier 'plugin.xml', il faut juste définire les points d'extensions 
qui existent pour le plugin Coloane, ici nous avons avons décaler deux points
d'extensions. En effet les contributions ne sont obligées d'implementer à la fois 
les fonctions d'importatios et d'exportations.

AJOUTER LES LIGNES DE CODES


\subsection{imports.exsd et exports.exsd}
Dans les fichiers 'imports.exsd' et 'exports.exsd', on définie la grammaire des 
points d'extensions. Plus présisement, le nom des attributs, et leur type.
\begin{itemize}
  \item id: C'est l'identifiant du points d'extensions.
  \item name: C'est le nom du nom du point d'extension
  \item class: C'est la classe que devrons implements les extensions
\end{itemize}

\subsection{MANIFEST.MF}
Dans le fichier 'MANIFEST.MF', en dois définire les packages que l'on doit exporter 
pour que les extensions, puissent implementer les fonctions d'importation et d'exportation.
Dans notre cas on exporter les packages suivants:
\\
Export-Package: fr.lip6.move.coloane.core.exceptions,\\
 fr.lip6.move.coloane.core.interfaces,\\
 fr.lip6.move.coloane.core.main,\\
 fr.lip6.move.coloane.core.motor.formalism,\\
 fr.lip6.move.coloane.core.ui.model\\
\\
qui definissent les exceptions que dervons lever les extensions , ainssi que les interfaces 
à implementer, et les methodes à utiliser pour créer un IModelImpl, etc...

\newpage

\section{Réalisation d'une extension}


\end{document}

