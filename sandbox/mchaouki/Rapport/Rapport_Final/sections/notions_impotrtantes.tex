\section{Notions impotrtantes}
Nous allons pr�senter quelques notions, avant de passer � la pr�sentation de 
Coloane, et aux m�canismes mis en jeu pour la conception et � la r�alisation 
d'un point d'extension.

\subsection{Plugin}
Un plugin, est un programme qui s'ajoute � une application principale, pour lui 
offrire de nouvelles fonctionnalit�s. Il existe plusieurs applications qui 
proposent d'ajouter des plugins. Par exemples, le navigateur internet 
"Mozilla FireFox", propose plusieurs plugins allant du simple traducteur, aux 
lecteur de MP3 int�gr� au navigateur. Tous ces plugins sont issus des 
contributions de developpeurs.

\subsection{Extensions}
Une extension peut �tre vue comme un plugin limit�. En faite, une extension, 
permet d'ajouter de nouveaux services, � une fonctionnalit� d�j� pr�sente dans 
une application. Reprenons notre pr�c�dent exemple sur le navigateur internet 
"Mozilla FireFox". Comme il a �t� dit plus haut, "Mozilla FireFox" proposent 
des plugins qui permettent de traduire des pages, dans une langue sp�cifique. 
Il est possible, si les plugins le permettent, d'ajouter de nouvelle langue au 
traducteur, via les contributions d'autres developpeurs, si ceux--ci developpent 
un extensions qui s'ajoute aux plugins existant.

\subsection{Points d'extensions}
Nous avons vue, qu'une extension, est un service qui venai s'ajouter aux 
fonctionnalit�s, d'une application (ou d'un plugin). Pour qu'une extension 
puisse s'ajoute � une application (ou � un plugin), il faut que celle--ci 
d�clare un point d'extension. La d�claration de points d'extensions, permet 
ainssi, � une application d'�tre extensible, et d'offrire de nouveaux services, 
venant d'autres contributions. Donc, un point d'extension, permet d'indiquer, 
aux extensions o� elle doivent venire se greffer, pour qu'elle puisse enrichire 
une application (ou un plugin).