\section{Coloane}

\subsection{Presentation de Coloane}
Coloane est un plugin, pour Eclipse, qui permet de mod�liser, les r�seaux de 
Petri. L'une de ses philosophies est d'�tre multi--platforme. Le moteur 
physique, utilise un format non standard, le CAMI, pour travailler, sur ces 
r�seaux de Petri. 

\subsection{Evolutions de Coloane}
L'une des �volutions tr�s impotrantes, � mettre en place pour Coloane, c'est 
d'offrire la possiblit� d'utilsier diff�rents formats de repr�sentations de 
r�seau de Petri, permettant ainssi l'�change de diff�rents formats de fichiers.

En effet, aujourd'hui, l'inconveniant majeur de Coloane commme nous l'avons 
�crit plus haut, c'est qu'il n'utilisent qu'un format, non standardiser le: 
CAMI. Or, il existe plusieurs formats pour la repr�sentation des r�seaux de 
Petri, dont un principale en cours de normalisation le: PNML.

\subsection{Limites li�s aux l'evolutions de Coloane}
Aujourd'hui, il est biens�re possible de faire �voluer Coloane, mais cela 
implique d'entrer � l'int�rieur du code, et de le modifier directement pour 
ajouter les �volutions voulus.

Or, le fait d'entrer dans le code, de Coloane, qui est actuelement stable, est 
un facteur important d'apparition de bugs. De plus, si plusieurs d�veloppeurs 
souhaitent contribuer au projet Coloane pour le faire �voluer, le code risque 
vite d'�tre confus, sans r�gles etablies.

Il serait donc, int�resant de pouvoire faire �voluer Coloane, sans toucher 
au code. Le fait de faire �voluer Coloane correspondrait, � ajouter de 
nouvelles fonctionnalit�es ou de nouveaux services. Or, ajouter des 
fonctionnalit�es ou services, c'est ajouter un plugin ou une extension.

Par cons�quant, si l'on veut faire evoluer Coloane, il faudrait 
songer � utiliser des plugins qui se mettent � coter de lui, ou des extensions 
qui viennent se greffer � lui.