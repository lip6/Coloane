\section{Pr�cisions sur les points d'extensions de Coloane}
Nous allons utiliser des points d'extensions, comme nous l'avons expliqu�s pr�cedement, 
pour permettre aux contributions d'offrire leur services d'importations et 
d'exportations. Utiliser des points d'extensions impliquent de pr�parer Coloane. Pour cela, 
il faut ajouter quelques lignes dans le fichier \textbf{plugin.xml} pour d�finire les points 
d'extensions, cr�er les fichiers \textbf{imports.exsd} et \textbf{imports.exsd } dans un 
repertoire \textbf{schema/} qui d�finissent la grammaire des points d'extensions, et 
modifier le fichier \textbf{MANIFEST.MF} pour lui sp�cifier les packages � exporter et 
qui sont suc�ptibles d'�tre utilis�s par les extensions.

Une remarque importante, Eclipse offre une int�rface graphique tr�s agr�able : 
PDE (Plug-ins Development Environment), qui permet  la conc�ption et la definition de points d'extensions sont 
diffucl�es.

\subsection{\textbf{plugin.xml}}
Dans le fichier \textbf{plugin.xml}, il faut juste d�finire les points d'extensions 
qui existent pour le plugin Coloane, ici nous avons avons d�cal� deux points
d'extensions: \textit{Imports} et \textit{Exports}. En effet les contributions ne sont oblig�es d'implementer � la fois 
les fonctions d'importatios et d'exportations.

Voicie, les deux lignes ajout�es dans le fichier \textbf{plugin.xml}:
\begin{verbatim}
<extension-point id="exports" name="Exports" schema="schema/exports.exsd"/>
<extension-point id="imports" name="Imports" schema="schema/imports.exsd"/>
\end{verbatim}

\subsection{\textbf{imports.exsd} et \textbf{exports.exsd}}
Dans les fichiers \textbf{imports.exsd} et \textbf{exports.exsd}, on d�finie la grammaire des 
points d'extensions. Plus pr�cisement, le nom des attributs, et leur types. Cela permet 
d'indiquer comment utiliser ces points d'extensions. Dans notre cas nous avons trois attributs/
\begin{itemize}
  \item id: C'est l'identifiant du points d'extensions.
  \item name: C'est le nom du nom du point d'extension
  \item class: C'est l'int�rface que devrons implements les extensions
\end{itemize}
Noter, qu'il est tr�s facile, gr�ce au PDE (Plug-ins Development Environment), de cr�er ces fichiers, 
\textbf{imports.exsd} et \textbf{exports.exsd}. En effet, il n'y a qu'� remplire des champs, et le PDE, se 
charge de g�n�rer automatiquement, les fichiers attendus.

\subsection{MANIFEST.MF}
Dans le fichier \textbf{MANIFEST.MF}, on doit d�finire les packages, que l'on doit exporter 
pour que les extensions puissent y avoir acc�s, afin d'impl�menter les fonctions 
d'importation et d'exportation.

Voicie, un extrait du fichier \textbf{MANIFEST.MF}, pr�sentant, dans notre 
cas les packages export�s, qui definissent les exceptions que dervons lever les extensions, ainssi que les interfaces 
� implementer, et les methodes � utiliser pour cr�er un IModelImpl, etc...

\begin{verbatim}
Export-Package: fr.lip6.move.coloane.core.exceptions,
 fr.lip6.move.coloane.core.interfaces,
 fr.lip6.move.coloane.core.main,
 fr.lip6.move.coloane.core.motor.formalism,
 fr.lip6.move.coloane.core.ui.model
\end{verbatim}