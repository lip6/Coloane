\documentclass{beamer}

\usepackage[french]{babel}
\usepackage{ucs}
\usepackage[utf8x]{inputenc}
\usepackage{graphicx}
\usepackage{hyperref}
\usepackage{tikz}
\usetikzlibrary{positioning,calc,shapes.multipart,shapes.geometric,arrows,fit,petri,automata,matrix}

\title{Coloane}
\subtitle{Environnement et Architecture}
\author[Clément Démoulins]{Clément Démoulins \\ \texttt{clement.demoulins@lip6.fr}}
\institute{Lip6}
\date{7 décembre 2011}

% Theme
%\usetheme{Hannover}
%\usetheme{Bergen}
\usetheme{Warsaw}

\useoutertheme{infolines}
%\useoutertheme{miniframes}

% logo
\titlegraphic{\center
	\parbox{2cm}{\includegraphics[width=2.0cm]{palmier}}
	\hspace{20pt}
	\parbox{1.3cm}{\includegraphics[width=1.3cm]{lip6}}
}

\usebeamercolor{block title}
\usebeamercolor{block title example}
\usebeamercolor{block title alerted}
\colorlet{Green}{block title example.bg}
\colorlet{Red}{block title alerted.bg}

\begin{document}

\frame{\titlepage}



\begin{frame}{Environnement}
\begin{block}{Java {\footnotesize $\geq 5$}}
\begin{itemize}
\item Langage de base
\end{itemize}
\end{block}
\begin{block}{Eclipse {\footnotesize $\geq 3.6$ Helios}}
\begin{itemize}
\item Plateforme d'exécution
\item Plateforme de développement
\item Utilisation du plugin \texttt{Graphical Editing Framework} (GEF)
\item Utilisation du plugin M2Eclipse
\end{itemize}
\end{block}
\begin{block}{Maven {\footnotesize $\geq 3$} \& Plugin Tycho}
\begin{itemize}
\item Gestionnaire de \texttt{build}
\item Gestionnaire de \texttt{packaging}
\end{itemize}
\end{block}
\end{frame}



\begin{frame}{Infrastructure}
\begin{exampleblock}{\texttt{coloane.lip6.fr}}
\begin{itemize}
\item Dépôt Subversion
\item Gestionnaire de ticket \& Wiki (\texttt{Trac})
\item Dépôts Eclipse (\texttt{Update-site})
\end{itemize}
\end{exampleblock}
\begin{exampleblock}{\texttt{teamcity-systeme.lip6.fr}}
\begin{itemize}
\item Platforme d'intégration continue (\texttt{TeamCity})
\item Dépôt d'artefact Maven (\texttt{Nexus})
\item Gestionnaire métriques \& tests \& qualité (\texttt{Sonar})
\end{itemize}
\end{exampleblock}
\end{frame}



\begin{frame}{Architecture}{Composants}
\begin{center}
\begin{tikzpicture}[
	node distance=5pt,
	element/.style={draw,rectangle,rounded corners,minimum height=20pt,thick},
	every text node part/.style={align=center},
]

\node [element,minimum width=60pt,draw=Green,fill=Green!20] (core)
	{Core};
\node [element,minimum width=80pt,draw=Green,fill=Green!20,right=of core] (interfaces)
	{Interfaces};
\node [element,dashed,inner sep=2pt,draw=Green,fit=(core) (interfaces)] (middleline) {};

\node [element,fill=black!20,draw=black!75,minimum width=152pt,below=of middleline] (eclipse)
	{Eclipse framework $+$ GEF};

\draw [very thick,draw=Red] ($(middleline.north west) + (14pt,60pt)$) arc (-180:0:6pt) arc (0:-90:6pt) -- node[color=Red,above,rotate=90] {imports} ($(middleline.north west) + (20pt,0)$);
\draw [very thick,draw=Red] ($(middleline.north west) + (34pt,60pt)$) arc (-180:0:6pt) arc (0:-90:6pt) -- node[color=Red,above,rotate=90] {exports} ($(middleline.north west) + (40pt,0)$);
\draw [very thick,draw=Red] ($(middleline.north west) + (54pt,60pt)$) arc (-180:0:6pt) arc (0:-90:6pt) -- node[color=Red,above,rotate=90] {formalisms} ($(middleline.north west) + (60pt,0)$);
\draw [very thick,draw=Red] ($(middleline.north west) + (74pt,60pt)$) arc (-180:0:6pt) arc (0:-90:6pt) -- node[color=Red,above,rotate=90] {examples} ($(middleline.north west) + (80pt,0)$);
\draw [very thick,draw=Red] ($(middleline.north west) + (94pt,60pt)$) arc (-180:0:6pt) arc (0:-90:6pt) -- node[color=Red,above,rotate=90] {apis} ($(middleline.north west) + (100pt,0)$);
\draw [very thick,draw=Red] ($(middleline.north west) + (114pt,60pt)$) arc (-180:0:6pt) arc (0:-90:6pt) -- node[color=Red,above,rotate=90] {tools} ($(middleline.north west) + (120pt,0)$);
\draw [very thick,draw=Red] ($(middleline.north west) + (134pt,60pt)$) arc (-180:0:6pt) arc (0:-90:6pt) -- node[color=Red,above,rotate=90] {reports} ($(middleline.north west) + (140pt,0)$);
\end{tikzpicture}
\end{center}
\end{frame}



\begin{frame}{Extensions existantes}
\begin{itemize}
\item \textbf{Formalismes} : PT Net, CPN, Reachibility Graph, Decision Diagram, Cosmos, GSPN, ITS, TPN
\item \textbf{Imports/Exports} : CAMI, PNML, ITS, ROMEO, TINA
\item \textbf{Imports} : PROD, SGROMEO
\item \textbf{Exports} : DOT, GML, PGF, SVG
\item \textbf{Outils} : Crocodile, Dot layout
\item \textbf{APIs} : AlligatorApi
\end{itemize}
\end{frame}



\begin{frame}{Architecture}{Core}
\footnotesize
\begin{itemize}
\item \textbf{core.extensions} Chargement et gestion de la plus part des points d'extensions
\item \textbf{core.formalisms.*} Chargement et gestion des formalismes
\item \textbf{core.models.*} Classes de la partie Modèle du MVC pour l'éditeur \texttt{ColoaneEditor}
\item \textbf{core.session} Un objet Session est lié à chaque \texttt{ColoaneEditor}
\item \textbf{core.ui.commands.*} Controlleurs
\item \textbf{core.ui.editpart} Vues
\item \textbf{core.ui.figures.*} Classes se chargeant du dessin des nœuds, arcs, notes, …
\item \textbf{core.ui.files} Gestion du chargement et de l'écriture des modèles dans le workspace
\item \textbf{core.ui.properties.*} Affichage des propriétés d'un modèles dans la vue \texttt{Properties} d'Eclipse
\item \textbf{core.ui.views} Navigateur de modèles
\item \textbf{core.ui.wizard} «Boîte de dialogue évoluée» pour l'export, l'import et la création de modèles
\end{itemize}
\end{frame}



\begin{frame}{Liens utiles}
\begin{itemize}
\item \url{https://coloane.lip6.fr/trac/} Trac de Coloane
\item \url{https://teamcity-systeme.lip6.fr/} TeamCity
\item \url{https://teamcity-systeme.lip6.fr/nexus/} Nexus
\item \url{https://teamcity-systeme.lip6.fr/sonar/} Sonar
\item \url{https://coloane.lip6.fr/trac/wiki/perso/demoulins} Information technique pouvant être utiles
\item \url{http://www.eclipsezone.com/eclipse/forums/c5605.html} Forum avec des informations liées au développement de plugin Eclipse
\item \url{http://keulkeul.blogspot.com/} Tutoriels
\item \url{http://www.psykokwak.com/blog/index.php/tag/gef} Tutoriels
\item \url{http://wiki.eclipse.org/Category:Tycho} Wiki du plugin Tycho
\end{itemize}
\end{frame}



\begin{frame}{Mécanisme d'extension}
\begin{block}{Point d'extension}
	\begin{itemize}
	\item Système d'extension d'Eclipse
	\item Description XML
	\item Spécification d'attributs simple ou de code
	\end{itemize}
\end{block}
\begin{exampleblock}{Example : exports}
	\begin{itemize}
	\item Identifiant
	\item Nom
	\item Wizard
	\item Extension
	\item Classe implémentant l'interface \texttt{IExportTo}
	\item Liste des formalismes supportés
	\end{itemize}
\end{exampleblock}
\end{frame}



\begin{frame}{Build de Coloane}
\begin{columns}[t]
	\begin{column}{6cm}
		\begin{block}{Développeur}
			\begin{enumerate}
			\item Checkout de : core, interfaces et des extensions
			\item (Optionnel) Récupérer les dépendances de certaine extension, \textit{Run As/Maven generate-sources}
			\item Démarrer Coloane, \textit{Run Eclipse Application}
			\end{enumerate}
		\end{block}
	\end{column}
	\begin{column}{5cm}
		\begin{alertblock}{Intégration continue}
			\begin{enumerate}
			\item Checkout du trunk complet de Coloane
			\item Build avec la commande \texttt{mvn package}
			\item Un \texttt{Update-site} est créé dans le dossier /trunk/update-site/target/repository/
			\end{enumerate}
		\end{alertblock}
	\end{column}
\end{columns}
\end{frame}



\end{document}