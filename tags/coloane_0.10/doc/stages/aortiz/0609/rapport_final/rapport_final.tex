\documentclass{article}
\usepackage[francais]{babel}
\usepackage[T1]{fontenc}
\usepackage[latin1]{inputenc}
\usepackage[left=2cm, right=3cm, top=2cm, bottom=2cm]{geometry}
\usepackage{pslatex}

\title{Rapport final}
\date{lundi 9 octobre 2006}
\author{Alexandre Ortiz}

\begin{document}

\maketitle

\tableofcontents

\newpage

\addcontentsline{toc}{section}{Introduction}
\section*{Introduction}
\par
Coloane est un plug-in pour Eclipse qui a �t� d�velopp�
par les M2 SAR de la promotion 2006, dans le cadre du
module Ing�ni�rie des Syst�mes R�partis.\\
Le but de mon stage a �t� de reprendre et de
proprifier le processus de d�veloppement et de
d�ploiement.

\section{Quelques mots sur Eclipse}
\subsection{Tout est projet}
\par
Un aspect d'Eclipse qui est, au d�part, assez d�routant, est
sa gestion des ``projets''.\\
Avant de faire ce stage, j'avais tendance � voir un projet
comme une entit� s�par�e en structures plus petites
pour les diff�rentes phases du projet (mod�lisation,
d�veloppement, tests, d�ploiement, ...).
\par
Avec Eclipse, ce sont ces diff�rentes structures qui sont
des projets (par exemple un projet pour l'�criture de tout
ou partie du plug-un, un pour les tests, un pour
le d�ploiement, ...).
\par
Il faut bien avoir cette repr�sentation en t�te pour comprendre
(un peu) Eclipse, et ne pas croire qu'un projet
au sens Eclipse du terme va comprendre tout le processus
du d�but � la fin.


\subsection{Un plug-in n'est pas un plug-in}

\subsubsection{Cas de Subclipse}
\par
Un plug-in, comme son nom l'indique en partie, est simplement
un bout de code (auquel sont souvent associ�s d'autres fichiers
comme les ic�nes, ...) qui se ``branche'' sur une
architecture d�j� en place.\\
Par exemple, si on veut explorer un d�p�t SVN �
travers Eclipse, il suffit d'installer le plug-in
Subclipse.\\
Apr�s installation de ce plug-in \textit{via} le site
d'update situ� � l'adresse http://subclipse.tigris.org/update\_1.0.x,
on s'apper�oit, en regardant dans Help->About Eclipse SDK->
Plug-in Details que ce n'est pas un plug-in qui a �t�
install�, mais trois :
\begin{itemize}
\item Subclipse
\item SVN Team Provider Core
\item SVN Team Provider UI
\end{itemize}
\par
Par contre, si on regarde dans les Feature Details,
il n'y a qu'une feature Subclipse qui a �t� install�e.

\subsubsection{Explications}
\par
Eclipse n'installe pas, \textit{via} son outil d'installation
et de mises-�-jour, des plug-ins mais des features (fonctionnalit�s).\\
Qu'est-ce qu'une feature ? Un groupement de plug-ins.\\
Quand on veut ajouter une fonctionnalit� � Eclipse, il est parfois
n�cessaire, pour des raisons de maintenabilit�,
de mettre diff�rentes parties de cette fonctionnalit�
(par exemple la GUI et le code fonctionnel du plug-in) dans plusieurs
plug-ins.\\
Dans ce cas, il serait absurde de pouvoir installer les plug-ins
de mani�re ind�pendante (par exemple installer seulement
la partie interface graphique).\\
C'est pour cela qu'Eclipse se limite � l'installation de features,
m�me si rien n'emp�che un utilisateur d'installer un plug-in
tout seul en le copiant simplement dans le r�pertoire plugins
d'Eclipse (mais l� Eclipse n'y est pour rien si l'utilisateur
fait quelque chose d'absurde).


\subsection{Eclipse et les insectes}
\par
La taille du projet Eclipse implique qu'il est difficile de
le garantir 100\% sans bugs.\\
Lors de mon stage, j'ai �t� confront� � des ``�v�nements
inexplicables'', qui ont �t� r�solus en red�marrant simplement
Eclipse.\\
Par exemple, Eclipse refusait de voir un projet comme un plug-in
et apr�s un red�marrage le projet �tait de nouveau vu comme
un plug-in.
\par
Les autres difficult�s que j'ai rencontr�es, qui semblaient
inexplicables au d�but, ont �t� solutionn�es (elles seront
d�taill�es dans la suite).

\subsection{Pr�requis}
\par
Le lecteur doit au moins savoir comment cr�er un projet et comment
installer un nouveau plug-in en utilisant le gestionnaire
d'installation d'Eclipse.
Dans la suite de ce document, je consid�rerai que l'automatic build
est d�sactiv� (``Project -> Build Automatically'').\\
Les manipulations qui sont indiqu�es se font dans la perspective
de PDE, sauf mention contraire.
\section{D�veloppement d'un plug-in avec Eclipse}
\subsection{PDE : plug-in Development Environment}
Eclipse, en tant que plate-forme tr�s fortement bas�e
sur la notion de plug-in, se devait d'avoir son
environnement de d�veloppement de plug-ins.\\
Cest chose faite avec PDE, qui fournit des mod�les
de projets pour toutes les phases du d�veloppement
et du d�ploiement d'un plug-in.\\
Il offre �galement des interfaces graphiques qui permettent de ne
pas avoir � modifier certains fichiers en mode texte (l'interface
graphique s'ouvre automatiquement quand on ouvre le fichier).


\subsection{Ecriture du plug-in}
\par
Un plug-in est constitu� :
\begin{itemize}
\item de code
\item de fichiers annexes (images, ...).
\end{itemize}
\par
Le but du projet de plug-in est d'offrir un environnement de
d�veloppement qui prenne en compte les besoins rencontr�s
lors de l'�criture d'un plug-in Eclipse, ainsi que de d�finir les
constituants du plug-in qui seront fournis dans la distribution.\\
PDE intervient ici en fournissant une interface graphique pour
la configuration du plug-in (fichiers plugin.xml, build.properties
et MANIFEST.MF).
\par
Les fichiers ``.properties'' d�finissent des variables (nom du plugin,
num�ro de version, extensions, chemins, ...) qui sont
utilis�s dans les .xml correspondant.

\subsection{Tests}
\par
C'est ici que le ``tout est projet'' entre en jeu, et permet
de r�pondre � la question : o� mettre les tests ?\\
Si on �crit les tests dans le m�me projet que le plug-in,
cela revient � les inclure dans la distribution du plug-in,
ce qui pr�sente peu d'int�r�t pour l'utilisateur final. Il
n'a en effet aucun besoin du code des tests pour ex�cuter
le plug-in.
\par
On pourrait jouer sur les r�gles de build pour ne pas inclure
les tests dans le plug-in distribu�, mais cette solution
nous force � nous �loigner de PDE pour la gestion
du projet.
\par
La solution que j'ai retenue est la cr�ation d'un nouveau
projet d�di� aux tests.\\
Ce projet a la structure classique d'un projet Eclipse. Le
seul �l�ment dont avons besoin est l'acc�s aux classes
du projet � tester. Pour cela, il suffit d'inclure le projet
� tester en d�pendance du projet de test.
\par
L'�criture des tests se fait ensuite de mani�re tout � fait
classique, en cr�ant une nouvelle classe de test dans le projet
de tests pour chaque classe � tester.

\subsection{D�ploiement}
\subsubsection{Encapsulation dans une feature}
\par
Comme nous l'avons vu plus haut, ce n'est pas un plug-in
mais une fonctionnalit� qui sera install�e (si l'on opte pour
une installation \textit{via} Eclipse).\\
Pour cela PDE propose le Feature Project, qui permet de
regrouper des plug-ins au sein d'une fonctionnalit�.\\
Les informations concernant ce type de projet sont
contenues dans le fichier feature.xml.\\
Comme pour les fichiers �quivalents dans le Plug-in Project,
PDE propose une interface graphique pour modifier le
fichier feature.xml.\\
De mani�re �quivalente au fichier plugin.properties, le
fichier feature.properties permet de d�finir des variables
qui seront utilis�es dans le feature.xml.

\subsubsection{Mise � disposition de la feature pour un Eclipse client}
\par
Maintenant que l'on dispose de notre feature, il faut g�n�rer les
fichiers n�cessaires pour le site d'update.\\
Un site d'update est un emplacement (sur le r�seau ou sur le syst�me
de fichiers) qui permet � l'outil d'installation d'Eclipse de t�l�charger
les fichiers n�cessaires � l'ajout d'une fonctionnalit�.\\

\section{Coloane}

\subsection{Presentation de Coloane}
Coloane est un plugin, pour Eclipse, qui permet de mod�liser, les r�seaux de 
Petri. L'une de ses philosophies est d'�tre multi--platforme. Le moteur 
physique, utilise un format non standard, le CAMI, pour travailler, sur ces 
r�seaux de Petri. 

\subsection{Evolutions de Coloane}
L'une des �volutions tr�s impotrantes, � mettre en place pour Coloane, c'est 
d'offrire la possiblit� d'utilsier diff�rents formats de repr�sentations de 
r�seau de Petri, permettant ainssi l'�change de diff�rents formats de fichiers.

En effet, aujourd'hui, l'inconveniant majeur de Coloane commme nous l'avons 
�crit plus haut, c'est qu'il n'utilisent qu'un format, non standardiser le: 
CAMI. Or, il existe plusieurs formats pour la repr�sentation des r�seaux de 
Petri, dont un principale en cours de normalisation le: PNML.

\subsection{Limites li�s aux l'evolutions de Coloane}
Aujourd'hui, il est biens�re possible de faire �voluer Coloane, mais cela 
implique d'entrer � l'int�rieur du code, et de le modifier directement pour 
ajouter les �volutions voulus.

Or, le fait d'entrer dans le code, de Coloane, qui est actuelement stable, est 
un facteur important d'apparition de bugs. De plus, si plusieurs d�veloppeurs 
souhaitent contribuer au projet Coloane pour le faire �voluer, le code risque 
vite d'�tre confus, sans r�gles etablies.

Il serait donc, int�resant de pouvoire faire �voluer Coloane, sans toucher 
au code. Le fait de faire �voluer Coloane correspondrait, � ajouter de 
nouvelles fonctionnalit�es ou de nouveaux services. Or, ajouter des 
fonctionnalit�es ou services, c'est ajouter un plugin ou une extension.

Par cons�quant, si l'on veut faire evoluer Coloane, il faudrait 
songer � utiliser des plugins qui se mettent � coter de lui, ou des extensions 
qui viennent se greffer � lui.
\section{Quelques probl�mes}
\subsection{D�veloppement}
\subsubsection{Chemins des d�pendances}
\par
C'est l'une des parties les plus importantes de la configuration
du projet.\\
La configuration de ces chemins se trouve � trois endroits
diff�rents.\\
Le premier est dans la configuration du projet et on y acc�de
en faisant un clic droit sur le nom du projet puis ``Build
Path -> Configure Build Path''.\\
Les deux autres sont accessibles \textit{via} l'interface
graphique de PDE qui s'ouvre quand on acc�de au fichier
plugin.xml, dans les onglets Runtime et Build.\\
Toute modification (m�me si elle para�t minime)
sur la configuration des chemins peut faire �chouer la construction
du projet.\\
La configuration actuelle a �t� faite avec (beaucoup d') essais
et (beaucoup d') erreurs et en regardant sur des exemples de plug-ins
existants.



\subsection{Utilisation}
\subsubsection{Affichage du menu Platform}
\par
L'ajout d'une nouvelle entr�e dans la barre de menus
Eclipse ne provoque pas automatiquement son affichage lorsque le plug-in
est charg� (ni lorsqu'une perspective associ�e au plug-in
l'est).\\
Les concepteurs d'Eclipse ont choisi ce comportement pour
�viter la prolif�ration de menus.\\
Pour afficher le menu Platform, c'est � l'utilisateur du plug-in
de sp�cifier qu'il souhaite que le menu s'affiche.\\
On active un menu avec ``Window -> Customize Perspective''.\\
Par contre, il faut faire attention � la perspective dans
laquelle on se trouve quand on modifie cette option. En effet,
si on active un menu, par exemple, dans la perspective Java,
il ne sera visible que dans cette perspective et sera
absent dans les autres.\\
Il faut donc que l'utilisateur se trouve dans la perspective de
Coloane lorsqu'il active le menu.
\par
Selon moi, 'id�al serait non pas d'avoir un nouveau menu, mais seulement
un ou plusieurs boutons pour configurer la connexion
� Framekit (m�me si cela provoque un changement par rapport � Macao).\\
Pour la connexion du mod�le en cours, une entr�e de plus dans
le menu contextuel du mod�le serait �galement plus ergonomique.

\subsubsection{Utilisation du menu Platform}
\par
Une fois r�gl� le probl�me de l'affichage, le menu pose
encore probl�me avec un fonctionnement assez al�atoire.\\
Les entr�es du menu ne produisent pas toujours l'effet escompt�,
avec comme constat que si �a a fonctionn� une fois lors d'une
session Eclipse, alors le menu ne posera plus de probl�me
dans la suite de la session.\\
Inversement, si, lorsque l'on clique sur une entr�e du menu,
rien ne se produit, alors le menu sera inutilisable pour
tout le reste de la session.


\addcontentsline{toc}{section}{Conclusion}
\section*{Conclusion}
Coloane, dans son �tat actuel, a �t� d�barass�  des biblioth�ques
inutiles (quelques centaines de kilo-octets contre pr�s de
37 M�ga-octets auparavant) et la structure du projet
permet de reprendre le d�veloppement sur des bases saines.

\end{document}