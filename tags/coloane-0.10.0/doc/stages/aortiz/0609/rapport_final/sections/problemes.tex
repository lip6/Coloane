\section{Quelques probl�mes}
\subsection{D�veloppement}
\subsubsection{Chemins des d�pendances}
\par
C'est l'une des parties les plus importantes de la configuration
du projet.\\
La configuration de ces chemins se trouve � trois endroits
diff�rents.\\
Le premier est dans la configuration du projet et on y acc�de
en faisant un clic droit sur le nom du projet puis ``Build
Path -> Configure Build Path''.\\
Les deux autres sont accessibles \textit{via} l'interface
graphique de PDE qui s'ouvre quand on acc�de au fichier
plugin.xml, dans les onglets Runtime et Build.\\
Toute modification (m�me si elle para�t minime)
sur la configuration des chemins peut faire �chouer la construction
du projet.\\
La configuration actuelle a �t� faite avec (beaucoup d') essais
et (beaucoup d') erreurs et en regardant sur des exemples de plug-ins
existants.



\subsection{Utilisation}
\subsubsection{Affichage du menu Platform}
\par
L'ajout d'une nouvelle entr�e dans la barre de menus
Eclipse ne provoque pas automatiquement son affichage lorsque le plug-in
est charg� (ni lorsqu'une perspective associ�e au plug-in
l'est).\\
Les concepteurs d'Eclipse ont choisi ce comportement pour
�viter la prolif�ration de menus.\\
Pour afficher le menu Platform, c'est � l'utilisateur du plug-in
de sp�cifier qu'il souhaite que le menu s'affiche.\\
On active un menu avec ``Window -> Customize Perspective''.\\
Par contre, il faut faire attention � la perspective dans
laquelle on se trouve quand on modifie cette option. En effet,
si on active un menu, par exemple, dans la perspective Java,
il ne sera visible que dans cette perspective et sera
absent dans les autres.\\
Il faut donc que l'utilisateur se trouve dans la perspective de
Coloane lorsqu'il active le menu.
\par
Selon moi, 'id�al serait non pas d'avoir un nouveau menu, mais seulement
un ou plusieurs boutons pour configurer la connexion
� Framekit (m�me si cela provoque un changement par rapport � Macao).\\
Pour la connexion du mod�le en cours, une entr�e de plus dans
le menu contextuel du mod�le serait �galement plus ergonomique.

\subsubsection{Utilisation du menu Platform}
\par
Une fois r�gl� le probl�me de l'affichage, le menu pose
encore probl�me avec un fonctionnement assez al�atoire.\\
Les entr�es du menu ne produisent pas toujours l'effet escompt�,
avec comme constat que si �a a fonctionn� une fois lors d'une
session Eclipse, alors le menu ne posera plus de probl�me
dans la suite de la session.\\
Inversement, si, lorsque l'on clique sur une entr�e du menu,
rien ne se produit, alors le menu sera inutilisable pour
tout le reste de la session.
